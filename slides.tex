\documentclass{beamer}
\usepackage{ragged2e}
\usepackage{graphics}
\usepackage{graphicx}
\usepackage{subfigure}
\usepackage{multicol}
\usepackage{color}
   \justifying

\mode<presentation>
{
       \definecolor{darkred}{rgb}{0.607,0.145,0.145}
       \definecolor{gold}{rgb}{1,0.73,0.24}
       \setbeamertemplate{background canvas}[vertical shading][bottom=darkred!10,top=gold!10]
       \usetheme{Warsaw}
       \usefonttheme[onlysmall]{structurebold}
}
% \usepackage{beamerthemesplit} // Activate for custom appearance
 \title[Variational Patterns]{Wavenumber Selection of Turing Patterns through Boundary Conditions}
\author[DPM]{David P. Morrissey}
% \institute[]{
% School of Mathematics\\
% University of Minnesota}
\date[17-May-12]{University of Minnesota, \\$3^{rd}$ of April 2014}
\subject{Wavenumber Selection}

\begin{document}
\frame{\titlepage}

\section{Cahn-Hilliard}

\begin{frame}
\frametitle{Heuristic Derivation}
Let $F[u]$ be the free energy of a volume $\Omega$ defined on a non-uniform property, $\phi$, of the medium.
\[
 F[\phi] = \int_\Omega f(\phi, \nabla\phi, \nabla^2\phi, \ldots) dx
\]
\pause
For an isotropic medium the free energy is invariant under the reflections $x_i \rightarrow -x_i$ and $x_i \rightarrow x_j$ so that f expands as
\[
 F[\phi] = \int_\Omega \left ( f_0(\phi) + \kappa_1 \nabla^2\phi + \kappa_2 |\nabla\phi|^2 \right) dx
\]
where $\kappa_1 = \frac{\partial f}{\partial\nabla^2\phi}$ and $\kappa_2 = \frac{\partial f}{\partial  |\nabla \phi|^2}$ are tensors given by the crystal structure of the medium.

\end{frame}

\begin{frame}
\frametitle{Heuristic Derivation}
Integrating by parts and dropping higher order terms we are left with an energy functional of the forming
\[
 F[\phi] = \int_\Omega \left ( f_0(\phi) + \frac{c}{2}|\nabla\phi|^2 \right) dx + \int_{\partial \Omega} \kappa_1 \nabla\phi\cdot n dS
\]
where $c = 2(\kappa_2 - \frac{d\kappa_1}{d\phi})$
\end{frame}

\begin{frame}
 \frametitle{Potentials}
Assume a parameter dependent phase transition past a bifurcating value from a ``random`` phase (homogeneuos $\phi$) to an ''ordered`` stable phase $\phi^*$.  
This can be modeled by a ''homogeneuos potential`` of the form
\[
 f_0(\phi ; \lambda) = \left ( (a(\lambda) - (b(\lambda) \phi)^2 )\right )^2
\]
for stable phases given as $\phi^*(\lambda) = \frac{\sqrt{a}}{b}$.


\pause
The variation of the energy with respect to the order parameter $\phi$ gives the total potential $\mu$
\[
\frac{\delta F}{\delta \phi} = \mu = \frac{\partial f_0}{\partial \phi} - c|\nabla\phi|^2
\]
\pause
so that the flux of the total potential is 
\[
 J = -D\nabla \mu
\]
for a diffusion constant D.
\end{frame}

\begin{frame}
\frametitle{Cahn Hilliard equation}
Applying a continuity equation we have Cahn-Hilliard as a coupled second order system:
\begin{eqnarray*}
 \phi_t &=& \nabla \cdot D\nabla\mu \\
 \mu &=& \frac{\partial f_0}{\partial \phi} - c|\nabla\phi|^2
\end{eqnarray*}
\pause
A general, phenomenalogical, 4th order, conservation law of a quantity which has a ''homogeneoues`` free energy and a ''gradient`` free energy
\[
 \phi_t = D\nabla^2[(1-\phi^2)^2 - \gamma \nabla^2 \phi ]
\]
\end{frame}

\begin{frame}
 \frametitle{Gradient Dynamics}
 Under the right boundary conditions the free energy functional serves as a Lyapunov functional.
 \[
  \frac{dF}{dt} = \int \frac{\delta F}{\delta \phi}\frac{d\phi}{dt} = \int \mu \phi_t
 \]
 \[
  \frac{dF}{dt} = D \int_\Omega \mu \nabla^2\mu = -D\int_\Omega |\nabla\mu|^2 + D\int_{\partial\Omega} \mu \frac{\partial \mu}{\partial n}
 \]
 \pause
 For homogeneoues Neumann or homogeneoues Dirichlet boundary conditions on the potential $\mu$, $F$ serves as a Lyapuov functional given that the original boundary term also vanishes
 \[
   \int_{\partial \Omega} \kappa_1 \nabla\phi\cdot n dS
 \]


\end{frame}



\section{Better model equation?}

\begin{frame}
 \frametitle{Swift-Hohenberg}
Originally developed as a model for convective instabilities in hydrodynamics, the Swift-Hohenberg equation
\[
u_t=-(\nabla^2-1)^2u + \mu u -u^3
\]
is a well known model for pattern forming systems.  The equation has equilibrium $u=0$ which for $\mu>0$ is linearly unstable.  
For particular boundary conditions, perturbations from the zero solution saturate to leading order as $u(x)=\sqrt{\frac{4\mu}{3}}\cos(kx + \varphi) + \mathcal{O}(\mu^{\frac{3}{2}})$
\end{frame}

\begin{frame}
 \frametitle{SH Gradient Dynamics}
Similar to Cahn Hilliard the Swift Hohenberg equation has a Lyapunov functional arising from the energy fuctional
\[
 F[u] = \frac{1}{2} \int_\Omega {[(\partial_x^2 + 1)u]^2 - \mu u^2 + \frac{1}{2}u^4 }dx.
\]
\pause
Taking a variation we find
\begin{eqnarray*}
 \frac{\delta F}{\delta u}v &=& \int_\Omega (u_{xx}v_{xx} + u_{xx}v + uv_{xx} + uv)dx + \int_\Omega (-\mu u + u^3)vdx \\
 &=& \int_\Omega (u_{xxxx} + 2u_{xx} + u -\mu u + u^3)vdx + BC \\
 &=& \int_\Omega -u_t vdx + [(u_{xx} + u)\phi_x - (u_{xxx} + u_x)\phi]_{\partial\Omega}
\end{eqnarray*}
\end{frame}

%[(u_{xx} + u)\phi_x - (u_{xxx} + u_x)\phi]_\Omega
\begin{frame}
\frametitle{Asymptotic solutions}
 Perturbation Theory gives us a family of stationary solutions parameterized by wavenumber
 \[
  u(x;k)=\sqrt{\frac{4\mu}{3}}\cos(kx ) + \mathcal{O}(\mu^{\frac{3}{2}}).
 \]
\pause
The only symmetry of Swift Hohenberg is translation invariance so that solutions can be paramterized by the wavenumber $k$ and phase $\varphi$.
\[
  u(x;k)=\sqrt{\frac{4\mu}{3}}\cos(kx + \varphi) + \mathcal{O}(\mu^{\frac{3}{2}}).
 \]
But, again the boundary conditions break this symmetry (Consider Neumann.)
\end{frame}

\section{Simplest boundary imposed pattern}
\begin{frame}
\frametitle{A Natural selection problem}
 We would like to understand the wavenumber and phase reationship of stationary solutions for diferent boundary conditions on the half line.  
 And identify cases of wavenumber selection and phase selection.
 \[
 -(\partial_x^2 + 1)^2u + \mu u - u^3 = 0  \;\text{  for  }\; x\in (0,\infty)
 \]
 With the boundary conditions
 \begin{eqnarray*}
  \gamma u''(0) + (1-\gamma)u'(0) + \gamma u(0) &=& 0 \\
  u'''(0) + \gamma u'(0) &=& 0
 \end{eqnarray*}
 SO that $\gamma = 0$ corresponds to Neumann boundary conditions and $\gamma = 1$ corresponds to ``transparent'' boundary conditions.

 


 
% Neumann boundary conditions $u'(0)=u'''(0)=0$ are a simple case of phase selection since the translation is fixed.
\end{frame}




\end{document}


